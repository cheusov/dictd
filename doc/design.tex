% design.tex -- 
% Created: Fri Feb 21 08:04:03 1997 by faith@cs.unc.edu
% Revised: Thu Feb 27 23:49:25 1997 by faith@cs.unc.edu
% Copyright 1997 Rickard E. Faith (faith@cs.unc.edu)
% This program comes with ABSOLUTELY NO WARRANTY.
% 
% $Id: design.tex,v 1.3 1997/02/28 05:09:54 faith Exp $
% 
%
\def\FileCreated{Fri Feb 21 08:04:03 1997}
\def\FileRevised{Thu Feb 27 23:49:25 1997}

\documentclass{article}
\usepackage{xspace}
\usepackage{ifthen}


\newcommand{\WS}{\mbox{\tt <WS>\ }}
\newcommand{\SP}{\mbox{\tt <SP>\ }}
\newcommand{\NL}{\mbox{\tt <NL>\ }}
\newcommand{\EOF}{\mbox{\tt <EOF>\ }}

% grammar support
\def\leftline{\topsep 0ex\partopsep 0ex\trivlist \raggedright \item[]}
\let\endleftline=\endtrivlist
\newcommand{\note}{\bigskip\par\noindent}
\newcommand{\prd}[3][]{\noindent\begin{leftline}\hspace{1em}{\it #2}\ ::=\ %
  \begin{minipage}[t]{.64\textwidth}\raggedright #3\end{minipage}%
    \ifthenelse{\equal{#1}{}}{}%
    {\begin{minipage}[t]{.54\textwidth}\raggedright #1\end{minipage}%
      \vspace{1ex}}%
    \end{leftline}}
\newcommand{\opt}[1]{[~#1~]}
\newcommand{\optlist}[1]{\leftbrace\ #1\ \rightbrace}
\newcommand{\lhs}[1]{{\it #1\/}}
\newcommand{\leftbrace}{\char'173}  % for cmtt10 font
\newcommand{\rightbrace}{\char'175}
\newcommand{\lit}[1]{{\tt #1}}

\newcommand{\dict}{\textsc{dict}\xspace}
\newcommand{\webster}{\textsc{webster}\xspace}

\begin{document}

\title{The Design and Implementation of the \dict Client/Server}
\author{Rickard E. Faith\thanks{faith@cs.unc.edu}}
\maketitle


\section{Introduction}


\subsection{Historical Background}

The \webster program has long been available on the Internet.  This program
uses a proprietary Webster's 7th edition dictionary which is available at
only a few sites.  Due to overwhelming load, most of these sites have shut
down access to the general Internet community.\footnote{We believe that
  some original Arpanet sites have a license for a database which allows
  serving the database to other sites, but we have been unable to confirm
  these reports.  We also believe that some NeXT machines contained a copy
  of the same or similar database and that some of the \webster servers on
  the Internet have served this database.} For example, prior to 1994,
generally accessible \webster servers were running at MIT, CMU, Indiana
University, and several other sites.  Since 1996, the only publicly
available server appears to be running at the University of Michigan,
although at least one other server is accessible via the WWW.

Fortunately, in the past few years, several freely-distributable
dictionaries and lexicons have become available on the Internet.  However,
these freely-distributable databases are not accessible via a uniform
interface, and are not accessible from a single site.  Further, they are
often small and incomplete individually, but would collectively provide an
interesting and useful database for providing definitions of English words.
Examples include the Free On-line Dictionary of Computing, the Jargon file,
the WordNet database, and MICRA's version of the 1913 Webster's Revised
Unabridged Dictionary.\footnote{URLs and more information about each of
  these databases are provided in a later section.} Several translating
dictionaries are also available.

\section{The \dict Protocol}

\prd{banner}{\lit{server} \WS \lhs{number} \WS \lhs{string} \WS
  \lhs{msg-id} \NL}

\note When a client connects, the server sends a \lhs{banner}.  \lhs{number}
is the highest protocol level supported by the server, \lhs{string} is an
arbitrary identification string, which may contain whitespace, and
\lhs{msg-id} is a timestamp.  The syntax of the timestamp corresponds to
the `msg-id' in RFC822 \cite{RFC822}, and \emph{must} be different each
time the the \dict server issues a banner greeting.  For example,
\begin{center}
\begin{verbatim}
<process-id.clock@hostname>
\end{verbatim}
\end{center}
The timestamp can be used by a client to provide secure transmission of a
shared secret in the manner described in RFC1939 \cite{RFC1939} for APOP
authentication.\note

\prd{error}{\lit{error} \WS \lhs{number} \WS \lhs{error-message} \NL}
\prd{status}{\lit{status} \WS \lhs{status-message} \NL}

\note After the \lhs{banner} is sent, the conversation is client-driven.
The server replies to the client requests with database information as
described below, and/or with an \lhs{error} or \lhs{status} string.  A list
of valid \lhs{number} codes is given below.  The \lhs{status} message may
take any form and implies that the request completed without error.  Some
server implementations may include a running total of commands,
definitions, or resources requested by the client.  This may be useful for
debugging or logging purposes.\note

\prd{ident}{\lit{client} \WS \lhs{number} \WS \lhs{string} \NL}
\prd{ident-response}{\lhs{error}}
\prd{ident-response}{\lhs{status}}

\note The \lhs{client-ident} from the client specifies the protocol version
that the client supports.  The server will reply with either an \lhs{error}
or a \lhs{status} message.  From this point forward, the client will drive
all communications.  Note that the \lhs{client-ident} is optional, thereby
allowing simple use with a \texttt{telnet} connection.\note

\prd{protocol}{\lit{protocol} \WS \lhs{number} \NL}
\prd{protocol-response}{\lhs{error}}
\prd{protocol-response}{\lhs{status}}

\note The client can request a protocol level change later in the
conversation.

This server capability is optional.  If the capability is not implemented,
\lhs{error} will be returned to the client in the same manner as for an
illegal keyword.  A server that does not implement this capability should
be fully compatible with the version 1 protocol described here.\note

\prd{auth}{\lit{auth} \WS \lhs{user-string} \WS \lhs{password-string} \NL}
\prd{auth-response}{\lhs{error}}
\prd{auth-response}{\lhs{status}}

\note The client can authenticate to the server using a username and
password.  For this version of the \lhs{auth} request, the
\lhs{password-string} will be computed as in the APOP protocol discussed in
RFC1939 \cite{RFC1939}.  Briefly, the \lhs{password-string} is the MD5
checksum of the concatenation of the \lhs{msg-id} and the ``shared secret''
(or password).  The shared secret is stored in the configuration files for
the client and the server.  Since the user does not have to type this
shared secret when accessing the server, the shared secret can be an
arbitrarily long passphrase.  Because of the computational ease of
computing the MD5 checksum, the shared secret should be significantly
longer than a usual password.

Authentication may make more dictionary databases available for the current
session.  For example, there may be some publicly distributable databases
available to all users, and other private databases available only to
authenticated users.  Or, a server may require authentication from all
users to minimize resource utilization on the server machine.

Authentication is an optional server capability and need not be
provided in a typical server.\note

\prd{show-db}{\lit{show} \WS \lit{databases} \NL}
\prd{show-db}{\lit{show} \WS \lit{db} \NL}
\prd{show-db-response}{\lit{databases} \WS \lhs{number} \NL \lhs{db-list}
  \lhs{status}}
\prd{db-list}{\lit{database} \WS \lhs{db-id} \WS \lhs{db-description} \NL
              \lhs{db-list}}
\prd{db-list}{}

\note The short description of the database list should be suitable for use
in a human-readable identification line of the form:
\begin{center}\tt
  From <short-description>:
\end{center}
which can be used to preface a retrieved definition.  A typical reply might
be as follows:

\begin{verbatim}
databases 3
database wordnet "WordNet 1.5 Database"
database jargon "Jargon File"
database foldoc "Free Online Dictionary of Computing"
status ok
\end{verbatim}

The descriptions given are from the databases themselves, and are stored
under the required ``!short!'' headword.  See below for a complete list of
required and optional headwords.\note

\prd{show-strat}{\lit{show} \WS \lit{strategies}}
\prd{show-strat}{\lit{show} \WS \lit{strat}}
\prd{show-db-response}{\lit{strategies} \WS \lhs{number} \NL \lhs{strat-list}
  \lhs{status}}
\prd{strat-list}{\lit{strategy} \WS \lhs{strat-id} \WS
  \lhs{strat-description} \NL \lhs{db-list}}
\prd{strat-list}{}

\note The strategies listed are all those available for use with the
\lhs{match} command.  Some strategies may not be useful with all databases.
A typical response might be as follows:

\begin{verbatim}
strategies 7
strategy exact "Match words exactly"
strategy prefix "Match prefixes"
strategy substring "Match substring occurring anywhere in word"
strategy regexp "Regular expression match"
strategy soundex "Match using SOUNDEX algorithm"
strategy levenshtein "Match words within an edit distance of one"
strategy reverse "Match given definition fragment"
status ok
\end{verbatim}

\prd{define}{\lit{define} \WS \lhs{word} \WS \lhs{db-id} \NL}
\prd{define}{\lit{define} \WS \lhs{word} \NL}
\prd{define-response}{\lhs{error}}
\prd{define-response}{\lit{definitions} \WS \lhs{number} \NL
                      \lhs{definition-list}
                      \NL \lhs{status}}
\prd{definition-list}{\lit{definition} \WS \lhs{word} \WS \lhs{db-id} \WS
                      \lhs{db-description} \WS \lhs{number} \NL
                      \lhs{bytes} \NL
                      \lhs{definition-list}}
\prd{definition-list}{}

\note The definitions may contain binary characters, so the \lhs{number} of
bytes in the definition is sent before the byte stream containing the
definition.  If the \lhs{db-id} is not specified, then the first dictionary
listed in the \lhs{db-list} is used.

An error may occur if the exact word is not found in the dictionary, or if
the \lhs{db-id} specified is not available.

\prd{match}{\lit{match} \WS \lhs{word} \WS \lhs{strat-id} \WS \lhs{db-id} \NL}
\prd{match}{\lit{match} \WS \lhs{word} \WS \lhs{strat-id} \NL}
\prd{match-response}{\lhs{error}}
\prd{match-response}{\lit{matches} \WS \lhs{number} \NL
                     \lhs{match-list}
                     \NL \lhs{status}}
\prd{match-list}{\lit{match} \WS \lhs{word} \WS \lhs{db-id} \NL
                 \lhs{definition-list}}
\prd{match-list}{}

\section{Other Dictionary Server Protocols}

\subsection{The \webster Protocol}

In the interest of preserving historical accuracy, the \webster protocol is
presented here as it was implemented for the University of Michigan server.
The original \webster protocol was able to access a thesaurus database and
probably had additional commands that are not listed in this summary.  Our
hope is to implement the \webster protocol as well as the \dict protocol in
the same server.  To allow a mode-less interface to the two protocols, we
have specified that the \dict protocol keywords are all lowercase.  Since
the \webster protocol requires all keywords to be in uppercase, this should
provide a suitable partitioning of the command namespace.  (For future
expansion, mixed-case commands are still available as a third
namespace.)\note

\prd{command}{\lhs{keyword} \opt{\lhs{space} \lhs{argument}} \NL}
\prd{\SP}{ASCII 32, a space}
\prd{\NL}{ASCII 13 ASCII 10, a CR LF sequence}
\prd{\NL}{ASCII 141, LispMachine NewLine character}
\prd{\EOF}{ASCII 128}
\prd{number}{String of ASCII digits representing a number}
\prd{word}{ASCII string representing a word}

\subsubsection{The Help Command}

\prd{help}{\lit{HELP} \NL}
\prd{help-response}{\lhs{help-document} \EOF}

\subsubsection{The Define Command}

\prd{define}{\lit{DEFINE} \SP \lhs{word} \NL}
\prd{define-response}{\lhs{wild-response}}
\prd{define-response}{\lhs{spelling-response}}
\prd{define-response}{\lhs{definition-response}}

\prd{wild-response}{\lit{WILD} \SP \lit{0} \NL}
\prd{wild-response}{\lit{WILD} \NL \lhs{word-list} \EOF}
\prd{word-list}{\lhs{numbered-word}}
\prd{word-list}{\lhs{numbered-word} \lhs{word-list}}
\prd{numbered-word}{\lhs{number} \SP \lhs{word} \NL}

\note A \lhs{wild-response} is given when the word to be defined contained
wildcard characters (`\%' which matches exactly one character, or `*' which
matches 0 or more characters).  The word numbers are unique and can be used
in place of the word itself in a \lhs{define} request.\note

\prd{spelling-response}{\lit{SPELLING} \SP \lit{0} \NL}
\prd{spelling-response}{\lit{SPELLING} \NL \lhs{word-list} \EOF}

\note The \lhs{spelling-response} is returned if the requested word could
not be found in the dictionary.  The list contains possible words which are
within a Levenshtein distance of one from the requested word.  The exact
\webster algorithm is unknown, but it appears that transposition, single
character deletion, single character addition, and single character
correction are all attempted).\note

\prd{definition-response}{\lit{DEFINITION} \SP \lhs{number}
  \NL \opt{\lhs{word-list}} \lhs{definition} \EOF}
\prd{definition}{Any amount of ASCII text}

\note A \lhs{definition-response} is given when an exact match occurs.  The
\lhs{number} given is the number of cross-references for that definition.
These cross-references are given in the \lhs{word-list} (which is empty if
there are zero cross-references given).


\subsubsection{The Complete Command}

\prd{complete}{\lit{COMPLETE} \SP \lhs{word} \NL}
\prd{complete-response}{\lhs{ambiguous-response}}
\prd{complete-response}{\lhs{completion-response}}

\prd{ambiguous-response}{\lit{AMBIGUOUS} \SP \lhs{number} \EOF}

\note The \lhs{ambiguous-response} is given if the requested word is the
prefix or wild-card match of zero or more than one other words in the
dictionary.\note

\prd{completion-response}{\lit{COMPLETION} \SP \lhs{word} \NL}

\note The \lhs{completion-response} is given if the requested word is the
prefix or wildcard match of exactly one word in the dictionary.  The
matching word is returned in the response.


\subsubsection{The Endings Command}

\prd{endings}{\lit{ENDINGS} \SP \lhs{word} \NL}
\prd{endings-response}{\lit{MATCHES} \SP \lit{0} \NL}
\prd{endings-response}{\lit{MATCHES} \NL \lhs{word-list} \EOF}

\note The endings command returns a prefix or wild-card match on the
requested word.  The notes specifically state that the numbers returned
cannot be used in a subsequent define command, but the implemented server
appears to allow this.  Note that this command is similar to ``DEFINE
word*''.


\subsubsection{The Spell Command}

\prd{spell}{\lit{SPELL} \SP \lhs{word} \EOF}
\prd{spell-response}{\lit{SPELLING} \SP \lhs{number} \NL}
\prd{spell-response}{\lit{SPELLING} \NL \lhs{word-list} \EOF}

\note If the requested word matches exactly one word in the dictionary, the
response number is one.  Otherwise it is zero.  Wildcards are not allowed
in spell requests.  If the word is not found, but other words are found
within a Levenshtein distance of one from the requested word, these words
are returned in a list (e.g., centre).

\subsubsection{Other Responses and Notes}

Additionally, the following error responses are possible:

\prd{error}{\lit{ERROR RECOVERABLE}\lhs{error-message}\NL}
\prd{error}{\lit{ERROR FATAL}\lhs{error-message}\NL}

\note If there is additional data in the response, the data is followed by
an EOF packet (ASCII 128).

After a fatal error, the server will terminate the connection.

\section{Required Database Headwords}

When databases are prepared for a server, they should contain several
headwords that provide information about the database.  Three headwords are
required:
\begin{description}
\item[!info!] Human-readable information about the original source of the
  data and about the copyright and licensing information that apply to the
  data.  The information should be sufficient to:
  \begin{enumerate}
  \item Locate the original source of the pristine, non-dict database (and
    to find a potential location of an updated version)
  \item Determine if the data can be freely served or distributed, or if
    restrictions on use or distribution apply to the data
  \item Locate the copyright holder so that any necessary permissions for
    use or distribution can be obtained
  \end{enumerate}
\item[!short!] A short (e.g., one line) description of the database.  This
  should be suitable for a credit line printed by a client of the form:
  \begin{center}\tt
    From <short-description>:
  \end{center}
\item[!long!] A longer description of the database, which will help the
  human database user to determine if a database is of interest.
\end{description}

Other headwords of the general form:
\begin{center}\tt
  !headword!
\end{center}
may be used to provide additional information, but these fields will not be
used by the initial server implementation.  All of these special fields
should be placed at the top of  any flat database files provided for use by
the server.  The goal here is to make it easy for a human to determine the
source, copyright, distribution, and use restrictions that apply to any
given database.

\section{Available Databases}

In general, dictionary databases that might be useful with \dict fall into
three classes:
\begin{itemize}
\item Databases which are in the public domain or which have licenses which
  specifically permit their use in a program such as \dict.
\item Databases which are accessible via the Internet or the WWW, but which
  have an unknown or ambiguous copyright or other licensing restrictions
  that make their use in \dict problematic.
\item Commercial databases which are available on CDROM, which \emph{might}
  be suitable for personal or educational use but which \emph{cannot} be
  redistributed using the \dict server.
\end{itemize}

Databases in the last two categories are listed here for informational
purposes only.  Many of these databases are probably not suitable for use
with \dict due to legal restrictions on their use and/or distribution.  We
do not have the legal resources available to determine which databases are
redistributable.

The only databases which we will provide for use with \dict are those that
are clearly in the Public Domain, or are otherwise distributable.  This may
mean that we have specific permission from the author or copyright owner to
redistribute the database with the \dict programs.

The example \dict server(s) and client(s) distributed in this package are
all distributed under the GNU General Public License or a less restrictive
license.

However, note that some of the databases that are available in \dict format
may have \emph{additional restrictions} on their use or distribution.  If
you want to redistribute a database, you should determine if this is
allowed by the database owner.  Do \emph{not} assume that the databases are
distributed under terms which are compatible with the \dict distribution
terms.


\section{Server Setup Files}

\subsection{Configuration File}

\prd{server-config-file}{\opt{\lhs{access}} \opt{\lhs{group-list}}
  \lhs{database-list}}
\prd{group-list}{\lhs{group} \opt{\lhs{group-list}}}
\prd{database-list}{\lhs{databse} \opt{\lhs{database-list}}}

\prd{access}{\lit{access} \lit{\{} \lhs{access-list} \lit{\}}}
\prd{access-list}{\lit{allow} \lhs{access-spec} \lhs{access-list}}
\prd{access-list}{\lit{deny} \lhs{access-spec} \lhs{access-list}}
\prd{access-list}{}
\prd{access-spec}{\lhs{user} \lit{@} \opt{\lhs{host-spec}}}
\prd{access-spec}{\lhs{group}\ lit{\#} \opt{\lhs{host-spec}}}
\prd{access-spec}{\lhs{host-spec}}
\prd{host-spec}{\lhs{host-addr-spec}}
\prd{host-spec}{\lhs{host-name-spec}}
\prd{host-addr-spec}{\lhs{number} \lit{.}}
\prd{host-addr-spec}{\lhs{number} \lit{.} \lhs{number} \lit{.}}
\prd{host-addr-spec}{\lhs{number} \lit{.} \lhs{number} \lit{.} \lhs{number}
  \lit{.}}
\prd{host-addr-spec}{\lhs{number} \lit{.} \lhs{number} \lit{.} \lhs{number}
  \lit{.} \lhs{number}}
\prd{host-name-spec}{\lhs{host-wild-spec}}
\prd{host-name-spec}{\lhs{string} \lhs{host-wild-spec}}
\prd{host-wild-spec}{\lit{.} \lhs{string} \opt\lhs{host-wild-spec}}

\prd{group}{\lit{group} \lit{\{} \lhs{user-list} \lit{\}}}
\prd{user-list}{\lhs{username} \opt{user-list}}

\prd{database}{\lit{database} \lhs{db-id} \lit{\{} \lhs{db-spec} \lit{\}}}
\prd{db-spec}{\lit{data} \lhs{filename}}
\prd{db-spec}{\lit{index} \lhs{filename}}
\prd{db-spec}{\lhs{access}}
\prd{db-spec}{\lhs{filter} \lhs{string}}
\prd{db-spec}{\lhs{prefilter} \lhs{string}}
\prd{db-spec}{\lhs{postfilter} \lhs{string}}

\subsection{Password File}

\prd{password-file}{\lhs{entry-list}}
\lhs{entry-list}{\lhs{entry} \opt{\lhs{entry-list}}}

\prd{entry}{\lhs{username}:\lhs{shared-secret}}


\section{Client Setup File}

\prd{client-config-file}{\lhs{server-spec-list} \opt{\lhs{server-group-list}}}
\prd{server-list}{\lhs{server-entry} \opt{\lhs{server-list}}}
\prd{server-group-list}{\lhs{server-group} \opt{\lhs{server-group-list}}}

\prd{server-entry}{\lhs{db-id} \lhs{server-spec} \lhs{shared-secret}}
\prd{server-spec}{\lhs{machine-name} \lit{:} \lhs{port}}
\prd{server-spec}{\lhs{machine-addr} \lit{:} \lhs{port}}

\prd{server-group}{\lhs{group-id} \lit{\{} \lhs{search-list} \lit{\}}}
\prd{search-list}{\lhs{db-id} \opt{\lhs{search-list}}}
\prd{search-list}{\lhs{group-id} \opt{\lhs{search-list}}}
\prd{search-list}{\lit{!} \opt{\lhs{search-list}}}

\end{document}
